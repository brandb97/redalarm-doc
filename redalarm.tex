\documentclass[12pt]{ctexart}

\usepackage{xltxtra}
\usepackage{amsmath}
\usepackage{booktabs}

\CTEXsetup[format={\Large\bfseries}]{section}

\pagestyle{plain}
\setcounter{tocdepth}{3}

\title{红警指南}
\author{yld}
\date{}

\begin{document}
	\maketitle
	\pagenumbering{gobble}
	\newpage
	\pagenumbering{arabic}
	\tableofcontents
	\newpage
	
	\section{建造价格}
		
	\newpage
	
	\section{建造依赖关系}
		\subsection{建造建筑依赖}
			依赖指能够建造建筑的前置条件,而不是建筑的运行条件。下文中$x \rightarrow y$表示$y$依赖于$x$
			\begin{align*}
				\text{兵营} &\rightarrow \text{战车工厂,雷达,维修厂}\\
				\text{战车工厂 + 雷达} &\rightarrow \text{实验室}\\
				\text{实验室} &\rightarrow \text{核电站,复制中心(苏)精炼厂(盟)}
			\end{align*}
			
		\subsection{兵种和防御依赖}
			\subsubsection{基础(兵营)}
			\begin{table}[h]
				\begin{center}
					\begin{tabular}{l|l}
						\toprule
						\midrule
						兵种 & 工兵,狗,工程师\\
						防御 & 围墙,地堡,防空炮\\
						\bottomrule
					\end{tabular}
					\label{tab:表1}
					\caption{基础阶段的兵种和防御}
				\end{center}
			\end{table}
			
			\subsubsection{中级(雷达$+$战车工厂)}
			\begin{table}[h]
				\begin{center}
					\label{tab:表2}
					\caption{中级阶段的兵种和防御}
				\end{center}
			\end{table}
			
			\subsubsection{高级(作战实验室)}
	\newpage
	
	\section{战术}
		\subsection{发展}
			\paragraph{抢箱子}
			\paragraph{采矿}
		\subsection{攻击}
			\paragraph{干扰}
			工程师,间谍,疯狂伊文,多功能步兵车,飞行兵,黑鹰战机,V3火箭,导弹船,超时空飞行兵都是非常有效的干扰源。
			
			工程师是前期主要的干扰源,使用步兵车和直升机都可以运送工程师。路径偷(进车后按Z)是最常见的用法。
			
			多功能步兵车可以有效对付任何空兵和步兵,同时可以装入步兵,磁暴步兵(雷电车)或者防空步兵(煤球车)使其更强大。主要用来干扰步兵和矿车。(待补充,tips:有人往车里装机器人,功能未知)
			\paragraph{进攻}
			使用坦克进攻。(待补充)
		\subsection{防御}
	\newpage
	
	\section{范例}
		时间:2023年6月6日
	
		地图:冰天雪地
		
		位置:右下角
		
		时间表:(待补充)

\end{document}