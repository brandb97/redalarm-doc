\documentclass[12pt]{ctexart}

\usepackage{xltxtra}
\usepackage{amsmath}
\usepackage{booktabs}

\CTEXsetup[format={\Large\bfseries}]{section}

\pagestyle{plain}
\setcounter{tocdepth}{3}

\title{红警指南}
\author{yld}
\date{}

\begin{document}
	\maketitle
	\pagenumbering{gobble}
	\newpage
	\pagenumbering{arabic}
	\tableofcontents
	\newpage
	
	\section{建造价格}
		
	\newpage
	
	\section{建造依赖关系}
		\subsection{建造建筑依赖}
			依赖指能够建造建筑的前置条件,而不是建筑的运行条件。下文中$x \rightarrow y$表示$y$依赖于$x$
			\begin{align*}
				\text{兵营} &\rightarrow \text{战车工厂,雷达,维修厂,船坞}\\
				\text{战车工厂 + 雷达} &\rightarrow \text{实验室}\\
				\text{实验室} &\rightarrow \text{核电站,复制中心(苏)精炼厂(盟)}
			\end{align*}
			
		\subsection{兵种和防御依赖}
			\subsubsection{基础(兵营)}
			\begin{table}[h]
				\begin{center}
					\begin{tabular}{l|l}
						\toprule
						兵种 & 工兵,狗,工程师\\
						防御 & 围墙,地堡,防空炮\\
						\bottomrule
					\end{tabular}
					\label{tab:表1}
					\caption{基础阶段的兵种和防御}
				\end{center}
			\end{table}
			
			\subsubsection{中级(雷达$+$战车工厂$+$船坞)}
			\begin{table}[h]
				\begin{center}
					\label{tab:表2}
					\caption{中级阶段的兵种和防御}
				\end{center}
			\end{table}
			
			\subsubsection{高级(作战实验室)}
	\newpage
	
	\section{战术}
		\subsection{兵种使用技巧}
		
		\subsection{基本战斗时间安排}
		\subsection{发展}
			\paragraph{抢箱子}
			使用Z构成环路并及时将产生的物品拉走
			\paragraph{采矿}
			应该迅速找到彩色矿并控制牛车采矿,在后期发展的过程中可以用兵营延伸或者拖动基地建立更多的矿场。
		\subsection{攻击}
			\paragraph{前期获得视野}
			首先应该用狗获得视野。
			
			一旦前期未获得视野,苏军可以用机器人,盟军可以用空军获得视野。双方都可以用多功能步兵车获得视野。
			\paragraph{前期抢油田}
			使用狗和地堡以及快速维修可以抢占油田。应该尽可能把兵营放置靠近油田方便使用地堡。
			\paragraph{干扰}
			工程师,间谍(苏)使用路径偷干扰,偷油田,基地,或重工。
			
			机器人进入牛车,杀死大兵。(苏)
			
			多功能步兵车和磁暴步兵可以杀死机器人。(苏)
			
			V3火箭和无畏级战舰轰炸。(苏)
			
			飞机轰炸。(盟,韩)
			
			航空母舰轰炸。(盟)
			
			延伸光棱塔。(盟)
			
			延伸巨炮。(法)
			
			后期藏幻影坦克。(苏)
			\paragraph{进攻}
			攻击的方式来源于对对手的观察。
		\subsection{防御}
		\subsection{快攻}
	\newpage
	
	\section{范例}
		时间:2023年6月6日
	
		地图:冰天雪地
		
		位置:右下角
		
		时间表:(待补充)

\end{document}